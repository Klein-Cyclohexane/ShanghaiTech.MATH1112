\documentclass{article}
\usepackage{ctex}
\usepackage{amsmath, amssymb}
\usepackage{geometry}
\geometry{a4paper, margin=1in}

\title{ShanghaiTech MATH1112: Linear Algebra \\ Fall 2025 Midterm}
\author{}
\date{}

\begin{document}

\maketitle

\section*{P1}
讨论在不同参数 $\lambda$ 下,线性方程组
\[
\begin{cases}
x_1 + x_2 - x_3 = 1 \\
x_1 + \lambda x_2 + 3x_3 = 2 \\
2x_1 + 3x_2 + \lambda x_3 = 3
\end{cases}
\]
何时有唯一解、无穷多解、无解,并在无穷多解时给出通解。

\section*{P2}
设 $V = \text{span}\{\mathbf{v}_1, \mathbf{v}_2, \mathbf{v}_3, \mathbf{v}_4\} \subseteq \mathbb{R}^4$,其中
\[
\mathbf{v}_1 = \begin{pmatrix} 1 \\ 2 \\ -1 \\ 1 \end{pmatrix}, \quad
\mathbf{v}_2 = \begin{pmatrix} 2 \\ 4 \\ -2 \\ 2 \end{pmatrix}, \quad
\mathbf{v}_3 = \begin{pmatrix} 3 \\ 7 \\ 2 \\ 0 \end{pmatrix}, \quad
\mathbf{v}_4 = \begin{pmatrix} 1 \\ 3 \\ 4 \\ -2 \end{pmatrix}.
\]
求 $V$ 的一组由 $\{\mathbf{v}_1, \mathbf{v}_2, \mathbf{v}_3, \mathbf{v}_4\}$ 中向量构成的基。

\section*{P3}
线性映射 $f: \mathbb{R}^n \to \mathbb{R}^m$, $\mathbf{v}_1, \dots, \mathbf{v}_k \in \mathbb{R}^n$, $\mathbf{v}_k = f(\mathbf{u}_k)$
\begin{enumerate}
    \item 证明: 若$\{\mathbf{v}_1, \mathbf{v}_2, \dots, \mathbf{v}_k\}$ 线性无关,则 $\{\mathbf{u}_1, \mathbf{u}_2, \dots, \mathbf{u}_k\}$ 也线性无关。
    \item 命题:"若$\{\mathbf{u}_1, \mathbf{u}_2, \dots, \mathbf{u}_k\}$ 线性无关,则 $\{\mathbf{v}_1, \mathbf{v}_2, \dots, \mathbf{v}_k\}$ 也线性无关。" 是否正确?是请给出证明,否请举反例说明。
\end{enumerate}

\section*{P4}
线性映射 $\phi: \mathbb{R}^4 \to \mathbb{R}^3$ 定义为:
\[
\phi([x_1, x_2, x_3, x_4]) = [x_1 + 3x_3 + x_4, \; 4x_2 + 5x_4, \; x_1 + 2x_2 + x_3].
\]
\begin{enumerate}
    \item 写出 $\phi$ 对应的矩阵 $A$,即 $\phi(\mathbf{x}) = A\mathbf{x}$。
    \item 求 $\ker \phi$ 与 $\operatorname{im} \phi$ 的维数。
\end{enumerate}

\section*{P5}
对 $n \in \mathbb{N}^+$ 及 $a_1, a_2, \dots, a_{n-1} \in \mathbb{R}$,有\(n\)阶方阵:
\[
M(a_1, a_2, \dots, a_{n-1}) =
\begin{pmatrix}
1 & a_1 & a_2 & \cdots & a_{n-2} & a_{n-1} \\
0 & 1 & 0 & \cdots & 0 & 0 \\
0 & 0 & 1 & \cdots & 0 & 0 \\
\vdots & \vdots & \vdots & \ddots & \vdots & \vdots \\
0 & 0 & 0 & \cdots & 0 & 1
\end{pmatrix}
\]
定义 $n$ 阶方阵集合:
\[
G = \{ M(a_1, a_2, \dots, a_{n-1}) \mid a_i \in \mathbb{R} \}.
\]
证明 $G$ 对于矩阵乘法与求逆均封闭,即 对$ \forall A, B \in G$,有 $AB \in G$ 且 $A^{-1} \in G$。

\section*{P6}
计算下述行列式:

$\begin{vmatrix}
1 & 1 \\
1 & 6
\end{vmatrix}
\hspace{3cm}
\begin{vmatrix}
2 & 0 & 1 \\
3 & 1 & 0 \\
1 & 2 & 2
\end{vmatrix}
\hspace{3cm}
\begin{vmatrix}
a_0 & -1 & 0 & 0 & \dots & 0 & 0 \\
a_1 & x & -1 & 0 & \dots & 0 & 0 \\
a_2 & 0 & x & -1 & \dots & 0 & 0 \\
\vdots & \vdots & \vdots & \vdots & \ddots & \vdots & \vdots \\
a_{n-2} & 0 & 0 & 0 & \dots & x & -1 \\
a_{n-1} & 0 & 0 & 0 & \dots & 0 & x
\end{vmatrix}$


\section*{P7}
设 $A, B$ 均为 $n$ 阶方阵,$[A \ B]$ 与 $\begin{bmatrix} A \\ B \end{bmatrix}$ 分别表示 $n \times 2n$ 与 $2n \times n$ 的分块矩阵。
\begin{enumerate}
    \item 证明:$\max\{\operatorname{rank} A, \operatorname{rank} B\} \le \operatorname{rank} [A \ B] \le \operatorname{rank} A + \operatorname{rank} B$。
    \item "$\operatorname{rank} [A \ B] = \operatorname{rank} \begin{bmatrix} A \\ B \end{bmatrix} $ "是否正确?若是,请给出证明;若否,请举反例说明。
\end{enumerate}

\section*{P8}
\[
A = \begin{pmatrix}
1 & 2 & 3 & \cdots & n-1 & n \\
0 & 2 & 3 & \cdots & n-1 & n \\
0 & 0 & 3 & \cdots & n-1 & n \\
\vdots & \vdots & \vdots & \ddots & \vdots & \vdots \\
0 & 0 & 0 & \cdots & n-1 & n \\
0 & 0 & 0 & \cdots & 0 & n
\end{pmatrix}.
\]
求 $A^{-1}$、$\sum\limits_{i=1}^n \sum\limits_{j=1}^n A_{ij}$($A_{ij}$ 指 $a_{ij}$的代数余子式)。

\section*{P9}
设 $A = (a_{ij})$ 是 $(n+1) \times n$ 矩阵,$\mathbf{x} = [x_1, x_2, \dots, x_n]$ 是未知数列向量,$\mathbf{b} = [b_1, b_2, \dots, b_{n+1}]^T$ 是常数列向量,且 $\operatorname{rank} A = n$。

证明:线性方程组 $A\mathbf{x} = \mathbf{b}$ 有解的充要条件是增广矩阵 $(A \mid \mathbf{b})$ 的行列式值为 $0$。

\end{document}